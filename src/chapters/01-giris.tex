\section{GİRİŞ}
LaTeX, aslında bir programlama dili değil, ama bir “belge yazma sistemi”. 
Yani Word gibi görüyorsun ama işin biraz daha kodlama gibi. 
Mesela metin, başlık, resim, tablo, matematik denklemleri vs. 
her şeyi özel komutlarla yazıyorsun.

Avantajı şu: yazdığın şey hep düzgün hizalanıyor, 
özellikle matematik ve bilimsel yazılarda sayfa düzeni çok güzel oluyor. 
Mesela denklem yazmak istersen tek bir satırla yapıyorsun:

\subsection{Temel Matematik Denklemleri}

\begin{enumerate}
\item \subsubsection{Enerji ve kütle ilişkisini gösteren ünlü denklem nedir?}
\[
E = mc^2
\]

\item \subsubsection{Dik üçgende kenarların ilişkisini ifade eden formül nedir?}
\[
\sqrt{a^2 + b^2} = c
\]
\[
\sqrt{a^2 + b^2} = c
\]

\item \subsubsection{Klasik mekanikte bir cismin kinetik enerjisi nasıl hesaplanır?}
\[
KE = \frac{1}{2} m v^2
\]

\item \subsubsection{1’den n’ye kadar olan sayıların toplamı nasıl bulunur?}
\[
\sum_{i=1}^{n} i = \frac{n(n+1)}{2}
\]

\item \subsubsection{Sonsuz integralin basit bir örneği nedir?}
\[
\int_0^\infty e^{-x} dx = 1
\]

\item \subsubsection{Genel ikinci dereceden fonksiyon nasıl yazılır?}
\[
f(x) = ax^2 + bx + c
\]

\item \subsubsection{2x2 boyutunda bir matris örneği verin:}
\[
\begin{pmatrix}
1 & 2 \\
3 & 4
\end{pmatrix}
\]

\end{enumerate}

ve LaTeX bunu otomatik güzel bir şekilde sayfaya koyuyor.

Bir de “paket” denilen ek modüller var. 
Mesela resim eklemek için graphicx paketini kullanıyorsun, 
tablolar için tabular var vs. Böylece her şeyi tek tek formatlamana gerek kalmıyor.

Çalıştırması da basit: bir .tex dosyası oluşturuyorsun, 
sonra \texttt{xelatex dosya.tex} komutuyla PDF’e çeviriyorsun. 
PDF çıktı alıyorsun, Word gibi ama daha stabil ve düzenli.

Kısacası LaTeX, metinleri kod gibi yazıp, 
sonunda çok düzgün, profesyonel görünümlü belgeler almak için kullanılan bir araç. 
Başta biraz uğraştırabilir ama alışınca çok rahat ediyorsun, aynı zamanda akademik 
çalışmalar için de gayet verimli bir araç hem güçlü matematik motoru hem de istikrarlı
yapısı sayesinde çoğu profesörün yıllar bildiği ve kullandığı bir ekosistemdir.
